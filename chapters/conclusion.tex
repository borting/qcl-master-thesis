%
%   Chapter Conclusion
%
%   Qing-Cheng Li (r01922024 at csie dot ntu dot edu dot tw)
%   R.O.C.103.07
%
\chapter{結論與未來展望}
\label{c:future}

\section{結論}
有別於過去的研究多半著重於於內容串流中找尋實體相關的文件,
本研究專注於於內容串流中尋找實體的特性,
提出了於基於樣式於內容串流中偵測實體特性的方式。
自內容串流中比對樣式,透過實體特性與樣式間的關聯,
快速地偵測文件是否包含特定的實體樣式。
由於偵測流程文件與文件是互相獨立的,
整個偵測的流程可平行化,具備可擴展性,
能夠處理模擬真實世界的內容串流。

本研究也為偵測特性的過程加入了樣式篩選與特性消歧義等步驟以提升偵測效能。
樣式篩選針對了樣式信心值、可信賴程度以及樣式歧義度三個面向進行篩選。
對樣式篩選越嚴格,則可以使用的樣式越少,因此召回率會下降。
例如對信心值進行篩選,精確率有提升,但整體效能仍下降;
對可信賴度進行篩選,精確率隨有提升但隨著門檻提高,太高的門檻反而使整體效能下滑。
而採用歧義度越高的樣式,則可採用的樣式總數增加,召回率有提升,而整體效能緩步下滑。

特性消歧義的步驟則是進行於樣式比對之後,
用以區辨樣式出現的地方是否無實體特性或有實體特性,是一個或多個實體特性。
透過引入實體類型資訊捨去不符合條件的特性,可在不改變召回率的情形下使系統的效能有顯著的提升。
針對特性的篩選實驗了四種篩選策略,由宏觀平均的角度來看,選擇正規化後正確率大於0.5的特性是最有幫助的。
最後再透過簡單貝氏分類器對特性進行存在或不存在的二元分類,可以顯著提升系統效能。
總體而論,結合引入實體類型資訊、選擇正規化後正確率大於0.5的特性後以簡單貝氏分類器進行分類可以得到最好的系統效能。

\section{未來展望}
就偵測流程本身,分析錯誤主要有兩類,第一類是選錯特性,
第二類是無法分辨有無特性存在,其中以第二類為主要的錯誤來源,
如何消彌這兩種錯誤以繼續改善偵測效能是未來的研究課題之一。

而本研究所使用之樣式,以及樣式與實體特性之關聯表來自於PATTY,
在樣式的覆蓋度以及與實體特性間的關聯度強弱受到一定程度的限制,
在未來希望可以自網路資源擷取樣式並與知識庫特性建立更細緻的關聯。

關於測試資料集,因為沒有正確的答案標記,只能透過YAGO來進行推測,
並假設所有的語句都是描寫該條目,希望未來可以針對文件,
甚至到句子等級的答案標記,可以進行更詳細的研究。
本研究目前是針對文件等級進行偵測,若有更細緻的答案標記,
或許可以進行句子等級的偵測。
更進一步,希望能夠知道句子中樣式描述的對象實體與其在句子中的位置,
才能更精確地偵測實體特性。
而Wikipedia的條目屬於較長的文章,也希望未來可以針對微網誌、社群留言等短訊息進行偵測。
使得偵測的面向更為廣泛以及更為細緻,甚至是自動化地對知識庫進行更新。

