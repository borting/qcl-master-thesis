%
%   Chapter Related Works
%
%   Qing-Cheng Li (r01922024 at csie dot ntu dot edu dot tw)
%   R.O.C.103.07
%
\chapter{相關研究與文獻}
\label{c:related}

% FIXME - 需要潤飾。
本章將介紹相關的研究:知識庫加速(Knowledge Base Acceleration)方面相關的研究,
及相關的資源:結構化的知識庫(Structural Knowledge Bases)與實體間的關係(Relations between Entities)。

%
%   KBA
%
\section{知識庫加速}

\cite{kba2012}與\cite{kba2013}整理了2012年、2013年TREC的知識庫加速競賽的總覽。
\cite{kba2012}追蹤60,000個被Wikipedia引用的網頁,
發現引用文章的時間與被引用文章的產生時間差中位數是一年,
顯示知識庫更新的速度遠落後於知識產生的速度,為了彌補這個差距,
連續兩年皆有累積引用推薦(Cumulative Citation Recommendation, CCR)任務,   % FIXME - 翻譯?
有一個特定的實體清單,並從內容串流中篩選出值得被引用的文件來建議知識庫的編輯者更新及維護知識庫。

% 內容串流描述

% 競賽結果簡述

% 圍繞「實體」為主的方法。
\cite{kba-hltoce}利用什麼方法,
而\cite{kba-msra}用的方法是,

% 除此之外...
另外,\cite{kba-entity-detection}

%
%   Structural KB
%
\section{結構化的知識庫}
% TODO

\cite{freebase}
\cite{dbpedia}
\cite{yago}

%
%   Pattern and Relation
%
\section{實體間的關係}

知識隨著時間變化是可能產生改變的,
\cite{relationsByTime} 提到有些實體間的關係是相對恆常的,例如《1Q84》的作者是村上春樹;
而有些關係則是隨著時間而變動的,例如美國的總統是布希,只有在某一段時間內是正確關係。
此研究將實體間的關係分為是否為恆常(Constant)或是否唯一(Unique),
人工挑選1,000組關係並利用時間、實體出現在特定時段內的頻率、文法等特徵對關係進行分類。

\cite{reverb} 建立了一套開放資訊擷取系統(Open Information Extraction System),
透過動詞表示的詞彙與句法限制,以<arg1, relation, arg2>的形式自動擷取出實體間的關係,
不限於人工標定的關係。

而Wikipedia中,每一篇文章,或稱條目,是描寫一個特定的實體,
\cite{wisenet} 利用Wikipedia文章中連至其他條目的連結作為實體,
以此擷取條目間,也就是實體間的關係。
由於同一種關係可能由不同的語句樣式(Patterns)來表達,
此研究應用了Wikipedia條目分類資訊、樣式的前後文來分類同義樣式。

\cite{patty2012}及\cite{patty}建立了一個名為PATTY的分類集(Taxonomy),
將用來表達關係的樣式進行分類,將同義樣式合併,並附上實體類別資訊,
將關係表達為「<type 1> \emph{Pattern} <type 2>」,Type 1和Type 2是實體的類別,
而Pattern則由單字(Words)、詞性標記(Part-Of-Speech Tags)、萬用字元(Wildcards)所組成,
例如「<person>'s [adj] voice * <song>」。   % FIXME - Example

PATTY自Wikipedia、New York Times抽取句子,
以史丹佛剖析器(Stanford Parser\footnote{http://nlp.stanford.edu/software/lex-parser.shtml})對句子建立剖析樹,
以YAGO、Freebase作為實體的字典,判斷若句子中有兩個實體,
則把兩實體間在樹上的最短路徑上的字句擷取出來作為樣式,
並進一步合併樣式為同義樣式集(Pattern Synset)。
比起\cite{reverb},PATTY可以抽取任意的關係,不被詞彙或句法所限制。

對每一個樣式,都存在一組支持集(Support Set),
由符合樣式的實體對(Pairs of entites)所組成。
透過支持集的大小,對每個樣式計算計算了一個信心值(Confidence),
將樣式中的實體類別從類別改為類別繼承架構中的更廣義的父節點類別,
以可以填入的實體對數作為分母,支持集作為分子,
得到的分數就是心信值,代表一個樣式的品質。

除了同義樣式集之外,此研究還做了關係釋義(Relation Paraphrasing):
給定一個來自知識庫的關係,判斷一個樣式是否可以描述此一關係。
PATTY釋義了225種DBpedia關係,包含127,811個樣式;25種YAGO關係,包含43,124個樣式。
其透過隨機選取1,000組釋義來評估,平均的精確度(Precision)是0.76$\pm$0.03。

本研究將嘗試利用PATTY提供的關係釋義來偵測實體特性是否存在於文件之中。   % FIXME - 收尾好像不是收的挺好

