%
%   Chapter Experiment
%
%   Qing-Cheng Li (r01922024 at csie dot ntu dot edu dot tw)
%   R.O.C.103.07
%
\chapter{實驗結果與分析}
\label{c:exp}

本章節進行了以樣式偵測特性的實驗,
以Wikipedia的文章作為模擬內容串流文件、
以YAGO的資料作為答案、
以PATTY提供的樣式對Wikipedia的文章進行偵測。
並介紹評估的標準以及對實驗結果的分析。

\section{測試資料集}
\label{s:dataset}

為了模擬內容串流文件,
本研究以Wikipedia的條目文章作為


% Wikipedia
% YAGO

% PATTY

\section{評估標準}
\label{s:eval}
在偵測系統的效能方面,
我們希望了解對每一個實體特性,
利用樣式自文章中偵測該特性的效能。
因此,以精確率(Precision)、召回率(Recall)、F1分數(F1 Score)進行評估。
精確率公式如式\ref{f:precision},召回率公式如式\ref{f:recall}。

\begin{equation}
    \label{f:precision}
    Precision = \frac{|\{relevant\ documents\}\cap\{retrived\ documents\}|}{|\{retrived\ documents\}|}
\end{equation}

\begin{equation}
    \label{f:recall}
    Recall = \frac{|\{relevant\ documents\}\cap\{retrived\ documents\}|}{|\{relevant\ documents\}|}
\end{equation}

對於一個實體特性,經過偵測之後會有被標為有此特性的文件與無此特性的文件。
其中,相關的文件(Relevant documents)即真正存在該特性之文件;
尋回的文件(Retrived documents)即偵測系統標記為擁有此特性之文件。
精確率評估在尋回的文件之中有多少文件真正存在該特性;
召回率評估在真正擁有該特性的文件中有多少被系統偵測到。

F1分數則是綜合評估精確率與召回率,計算方式如式\ref{f:f1},為精確率與召回率的調和平均。

\begin{equation}
    \label{f:f1}
    F_1\ Score = \frac{2\times Precision \times Recall}{Precision + Recall}
\end{equation}

除了評估個別特性的效能之外,
也以宏觀平均(Macro Average),以精確率為例,如式\ref{f:macro},召回率亦同,以及微觀平均(Micro Average),以精確率為例,如式\ref{f:micro},召回率亦同,評估整體偵測的效能。

\begin{equation}
    \label{f:macro}
    Macro\ Avg\ Precision=\frac{\sum_i^n precision_i}{n}
\end{equation}

\begin{equation}
    \label{f:micro}
    Mirco\ Avg\ Precision=\frac{\sum_i^n |\{relevant\ documents\}_i|\cap|\{retrived\ documents\}_i|}{\sum_i^n |\{retrived\ documents\}_i|}
\end{equation}


\section{實驗結果}
\label{s:result}

%\subsecion{}
%錯誤分析
