\begin{acknowledgementszh}
前年,我在普林斯頓,瀏覽愛因斯坦的《我所看見的世界》,得到了新的領悟。

這是一本非科學性的文集,專載些愛因斯坦在紀念會上啦、在歡迎會上啦、在朋友的葬禮中,他所發表的談話。

我在讀這本書時忽然發現愛因斯坦想盡量給聽眾一個印象:即他的貢獻不是源於甲,就是由於乙,而與愛因斯坦本人不太相干似的。

就連那篇亙古以來嶄新獨創的狹義相對論,並無參考可引,卻在最後天外飛來一筆,「感謝同事朋友貝索的時相討論。」

其他的文章,比如奮鬥苦思了十幾年的廣義相對論,數學部分推給了昔年好友的合作;這種謙抑,這種不居功,科學史中是少見的。

我就想,如此大功而竟不居,為什麼?像愛因斯坦之於相對論,像我祖母之於我家。

幾年來自己的奔波,作了一些研究,寫了幾篇學術文章,真正做了一些小貢獻以後,才有了一種新的覺悟:即是無論什麼事,得之於人者太多,出之於己者太少。

因為需要感謝的人太多了,就感謝天罷。
\end{acknowledgementszh}

%\begin{acknowledgementsen}
%I'm glad to thank\ldots 
%\end{acknowledgementsen}
